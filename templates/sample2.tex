\documentclass[a4paper,10.5pt]{jsarticle}
\usepackage[top=25mm,bottom=25mm,left=25mm,right=25mm]{geometry} % 余白を設定
\usepackage{titlesec} % タイトル設定
\usepackage{graphicx} % 図表用
\usepackage{caption} % キャプション調整
\usepackage{amsmath,amssymb} % 数式用
\usepackage{hyperref} % ハイパーリンク用
\hypersetup{
    colorlinks=true,
    linkcolor=black,
    filecolor=black,
    urlcolor=blue,
    citecolor=black
}
% ページの文字数・行数を指定するための関数
\makeatletter
\def\mojiparline#1{
    \newcounter{mpl}
    \setcounter{mpl}{#1}
    \@tempdima=\linewidth
    \advance\@tempdima by-\value{mpl}zw
    \addtocounter{mpl}{-1}
    \divide\@tempdima by \value{mpl}
    \advance\kanjiskip by\@tempdima
    \advance\parindent by\@tempdima
}
\makeatother
\def\linesparpage#1{
    \baselineskip=\textheight
    \divide\baselineskip by #1
}
% 段落間の余白とインデント設定
\setlength{\parindent}{2em} % 段落のインデント
\setlength{\parskip}{0pt} % 段落間の余白

% 見出しデザイン設定
\titleformat{\section}{\Large\bfseries\sffamily}{\thesection}{1em}{}
\titleformat{\subsection}{\large\bfseries\sffamily}{\thesubsection}{1em}{}

% 行数・文字数の設定

% ページ番号
\pagestyle{plain}

% 文書開始
\begin{document}

% 表紙
\begin{titlepage}
\centering
\vspace*{\fill}
{\Huge\bfseries 文書タイトル} \\[2cm]
	{\large 著者名} \\[0.5cm]
	\vspace*{\fill}
	{\large YYYY年MM月DD日} % 提出日
\end{titlepage}

% 目次
\pagenumbering{gobble} % ページ番号を非表示
\tableofcontents
\newpage
\pagenumbering{arabic} % ページ番号を表示
% 一行あたり文字数の指定
\mojiparline{40}
% 1ページあたり行数の指定
\linesparpage{30}

% 本文開始
\section{はじめに}
ここに本文を書きます。

\section{関連研究}
関連研究の内容を記述します。

\section{実験}
実験内容を記述します。

\section{結果}
結果を記述します。

\section{考察}
考察を記述します。

\section{まとめ}
まとめを記述します。

% 参考文献
\newpage

\bibliographystyle{junsrt}
\bibliography{cite}

% あとがき
\newpage
\section*{あとがき}
ここにあとがきを記述します。

% 謝辞
\newpage
\section*{謝辞}
ここに謝辞を記述します。

% 付録
\newpage
\section*{付録}
ここに付録を記述します。

\end{document}